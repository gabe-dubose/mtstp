\documentclass{article}
\usepackage{multicol}
\usepackage{xcolor}
\usepackage[left=1in, right=1in, top=1in, bottom=1in]{geometry}
\usepackage{indentfirst}

\title{Detailed Methods}
\author{DuBose and de Roode}
\date{20XX}

\begin{document}
\maketitle

\begin{multicols}{2}
    \section{Experiment Overview}

    \section{Milkweed Growing}
    \indent \emph{A. incarnata} and \emph{A. curassavica} seeds were purchased from Joyful Butterfly (Blackstock, SC, USA). To break cold dormancy, 
    seeds were placed in bags with sand and kept in a refrigerator at 4°C for two months prior to sowing. 
    Approximately two months before the start of the experiment, seeds were sown into Lambert LM-GPS germination soil and 
    placed in a temperature controlled greenhouse room (held between 25°C and 29.4°C). A. incarnata germination rates tend to be 
    low, so seed trays were topped with vermiculite to aid in moisture retention. Seedlings were fertilized with approximately 
    \textcolor{red}{X} PPM of Jack’s LX 15-5-15 with 4\% Ca and 2\% Mg fertilizer three times a week until the majority of plants grew two sets 
    of true leaves. All plants were then re-potted into Pro-mix BK25 soil, moved to a new temperature controlled room 
    (held between 25.6°C and 29.4°C), and fertilized three times a week as described above. Approximately one week before the 
    start of the experiment, plants were moved into the same greenhouse room that caterpillars were reared in (described below). 

    \section{Monarch Rearing}
    \indent Monarch butterflies were caught and labeled near St. Marks, Florida, U.S.A. (30°09′33″N 84°12′26″W) on \textcolor{red}{dd-mm-yyyy}. 
    Clear tape was placed on the abdomend of each butterfly and examined under a stereomicroscope to check for O. elektroscirrha spores. 
    Prior to mating season, wild-caught monarch butterflies were stored in glassine envelopes, placed in a refrigerator (14°C) to 
    induce a state of diapause, and were occasionally fed 10\% honey water. On \textcolor{red}{dd-mm-yyyy}, wild-caught monarchs were placed in 
    mating cages in \textcolor{red}{climate controlled growth chambers (get details).} After mating, female butterflies were individually placed 
    in separate cages in the same conditions and given A. curassavica for oviposition. After a given female was done laying eggs, 
    the plant was taken out of the growth chamber and placed in a temperature controlled greenhouse room (held between 23.3°C and 27.8°C) 
    for the eggs to hatch. 
    \\
    \indent F1 caterpillars were then reared on A. curassavica in the same greenhouse room previously described. After pupation, the silk 
    attached to the end of the pupal cremasters was used to hot glue the pupae to the lid of clear solo cups, which were then taken from 
    the greenhouse to the lab for eclosion. A piece of paper towel was placed in the bottom of cups to help absorb liquids produced by the 
    eclosion process. After eclosion, butterflies were placed in glassine envelopes and stored in a refrigerator as previously described.
    \\
    \indent On \textcolor{red}{dd-mm-yyyy} F1 butterflies not infected with \emph{O. elektroscirrha} and from different lineages were mated as previously 
    described in the F0 generation. F1 females were given either A. curassavica or A. incarnata for oviposition, and caterpillars were collectively 
    placed on their treatment plant species upon hatching. Care was taken to make sure caterpillars that had taken bites of the plant they were 
    oviposited on to were placed on the same milkweed species, Likewise, only caterpillars that had not taken any bites of the plant they were 
    oviposited on were placed on the other milkweed species (Figure 1). To reach the sample size needed for this experiment, we used F2 caterpillars 
    from two different lineages that did not share F0 or F1 ancestors. Treatments of parasite infection, plant species, and development stage were 
    randomly distributed to caterpillars from both lineages to minimize confounding due to genetic background.

    \section{\emph{O. elektroscirrha} innoculation}
    \indent Once caterpillars reached the third instar, parasite treatments were administered. \textcolor{red}{Details about OE spore acquisition here.} 
    \emph{O. elektroscirrha} infection was performed by placing caterpillars into petri dishes with a milkweed leaf disc corresponding 
    to their milkweed and infection treatment groups. Prior to placing milkweed leaf discs into the petri dishes, qualitative filter paper was placed 
    into petri dishes and wetted with deionized water to prevent leaf discs from drying out before the caterpillars could finish eating them. Milkweed 
    leaf discs (approximately 6.35 mm in diameter) were obtained by using a standard hole puncher to take sections of the leaf that included side 
    veins but not the midrib, as third instar caterpillars often do not eat leaf midribs. The hole puncher was cleaned using 75\% ethanol between 
    plants to minimize contamination of leaf discs with the other plants chemicals. For infected treatments, 100 O. elektroscirrha spores were counted 
    using a stereomicroscope and placed on the veins of the milkweed leaf discs that corresponded to the caterpillars treatment. Leaf discs fed to 
    uninfected caterpillars were punched at the same time but did not have parasite spores placed onto them.
    \\
    \indent After all petri dishes and leaf discs were prepared, caterpillars were taken from the plant they were feeding on and placed in a 
    petri dish with a leaf disc of their corresponding treatment. After caterpillars finished eating their whole leaf discs, they were put 
    on a new plant corresponding to their treatment plant species. Transparent plastic tubes with mesh nets on top were placed around all 
    plants to confine the caterpillars. All tubes were placed on the same shelf in the greenhouse room to minimize micro-climate variation. 

    \section{Developmental stage sampling and snap freezing}
    \indent To minimize changes in transcription due to cold stress, all caterpillars, pupae, and adults were flash frozen in liquid nitrogen 
    before storage in -80°C. Third instar caterpillars were frozen the day after parasite inoculation. Caterpillars were pulled from their 
    feeding plant and quickly placed into a sterile 2mL microcentrifuge tube that was then dipped in liquid nitrogen. Fifth instar 
    caterpillars were frozen in the same way, but were placed in sterile 5mL centrifuge tubes. The remaining samples were left on their plant 
    for pupation. Caterpillars that finished the plant were originally placed on were then placed on another plant of the same species. 
    One day after pupation, early pupa samples were placed in 5mL centrifuge tubes and frozen in liquid nitrogen. 
    \\
    \indent Three days after pupation, pupae for late pupa and adult samples were removed from their plant and taped to the lids of clear 
    solo cups using silk attached to the cremaster. This allows for better assessment of the early signs of O. elektroscirrha proliferation, 
    indicated by pupal darkening. In some cases, not enough silk detached from the plant, and tape was applied directly to the cremaster. 
    Solo cups were then placed on the bottom rack of the same shelf that the caterpillars were reared on, and shade was provided to avoid 
    burning by placing plastic trays above and to the southeast facing side of the shelf. A piece of paper towel was placed in the bottom of 
    the cups containing adult samples to absorb fluids produced during the pupation process. O. elektroscirrha-infected late pupa samples 
    were frozen in liquid nitrogen six to eight days after pupation, depending on when evidence of O. elektroscirrha proliferation was presented. 
    For each infected late pupa sample, an uninfected sample that fed on the same plant and had been a pupa for the same amount of time was also 
    frozen. Adult samples were frozen on the same day that they eclosed. Here, adults were removed from their solo cup and quickly placed in a 
    glassine envelope, which was then quickly placed in liquid nitrogen. 
    \\
    \indent All freezing took place in the same greenhouse room that the caterpillars were reared in, and no caterpillar left said room 
    throughout the duration of the experiment. After flash freezing in liquid nitrogen, samples were stored in a styrofoam cooler full of 
    dry ice until all freezing for that day was completed. This process took approximately one hour or less on any given day, so no sample was 
    on dry ice for more than an hour before being transferred to the -80°C freezer. 

\end{document}